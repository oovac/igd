\documentclass[11pt]{article}
\usepackage[margin=2.5cm]{geometry}
\usepackage[utf8]{inputenc}
\usepackage[T1]{fontenc}
\usepackage[english]{babel}

% Shared notation and symbols
% File: IGD_Notation_v1.tex
% Shared notation and symbol conventions for IGD papers.
%
% This file is intended to be \input in the preamble of each IGD LaTeX document.

\usepackage{amsmath,amssymb,amsfonts,amsthm}
\usepackage{bm}

% --- Sets, spaces, manifolds ---

% Hilbert space and state space
\newcommand{\Hilb}{\mathcal{H}}           % underlying Hilbert space
\newcommand{\States}{\mathcal{S}}         % space of physical states (density operators)
\newcommand{\Pure}{\mathcal{P}}           % manifold of pure states

% Parameter / model manifolds
\newcommand{\Mtheta}{\mathcal{M}_\theta}  % model (focus) manifold
\newcommand{\Mphys}{\mathcal{M}}          % generic state manifold

% Subsystems
\newcommand{\HA}{\Hilb_A}
\newcommand{\HB}{\Hilb_B}

% --- Geometry: metric, symplectic form, complex structure ---

\newcommand{\gij}{g_{ij}}                 % components of the information metric
\newcommand{\gmat}{g}                     % metric tensor
\newcommand{\om}{\omega}                  % symplectic 2-form
\newcommand{\Jc}{\mathcal{J}}             % (almost) complex structure on pure states
\newcommand{\Jop}{J}                      % antisymmetric operator implementing Hamiltonian flow
\newcommand{\Rop}{R}                      % symmetric operator implementing dissipative flow

% --- Energetic and entropic functionals ---

\newcommand{\Efun}{E}                     % energy functional on \Mtheta
\newcommand{\Sfun}{S}                     % entropy functional on \Mtheta

% --- Gradients and flows ---

\newcommand{\grad}{\nabla}                % gradient with respect to g
\newcommand{\dotth}{\dot{\theta}}        % time derivative of parameters
\newcommand{\ddt}{\frac{d}{dt}}          % total time derivative

% --- Information-theoretic quantities ---

\newcommand{\MI}{I}                       % mutual information
\newcommand{\SX}{S}                       % entropy (generic)
\newcommand{\KB}{k_{\mathrm{B}}}          % Boltzmann constant

\newcommand{\xiIGD}{\xi}                  % correlation / information length scale
\newcommand{\vIGD}{v_{\mathrm{IGD}}}      % information velocity (Lieb--Robinson-like)

% --- Misc ---

\newcommand{\Tr}{\mathrm{Tr}}
\newcommand{\id}{\mathbbm{1}}


\title{Information--Geometric Dynamics:\\ Axiomatic Foundation (v3.0)}
\author{Ordo Vacui and Collaborator}
\date{\today}

\begin{document}
\maketitle

\begin{abstract}
This document collects the axiomatic foundation of Information--Geometric
Dynamics (IGD) in a single, coherent form. It is intended to supersede
earlier drafts (v1.x, v2.x) by organizing the theory into a clear ontological
layer (facts about the world) and an epistemic/geometric layer (models,
focus manifolds and information geometry). The axioms A0--A10, together
with the IGD dynamical equation, provide a minimal but powerful basis from
which spacetime, gravity and complexity can be treated as emergent
information-geometric phenomena.
\end{abstract}

\tableofcontents

\section{Introduction and scope}
\label{sec:intro}

This v3.0 foundation aims to:
\begin{itemize}
  \item separate clearly the \emph{ontological} assumptions (what is taken
        as a fact about the structure of the physical world) from the
        \emph{epistemic/geometric} assumptions (how we model and coarse-grain
        our incomplete knowledge);
  \item state the axioms A0--A10 in a compact and non-redundant form,
        with inline commentary on their role;
  \item prepare the ground for the spacetime and gravity programme
        (theorems T1--T3) without presupposing them as axioms.
\end{itemize}

This document is designed to be read together with:
\begin{itemize}
  \item the IGD core dynamics and spacetime theorems (T1--T3) in a companion
        paper \emph{IGD\_Core\_Dynamics\_and\_Theorems\_v3.0};
  \item the model instantiations (TFIM, CFT, SYK) in
        \emph{IGD\_Applications\_and\_Models\_v1.0}.
\end{itemize}

\section{Ontological layer: information, systems and processes}
\label{sec:ontology}

In this section we collect axioms that we interpret as statements about
the structure of the physical world at the most abstract level: systems,
states, composition and fundamentally allowed transformations.

\subsection{A0: Quantum structure}
\label{subsec:A0}

Axiom A0 encodes the basic ``quantum'' character of the theory. It can be
formulated in terms of three operational principles:
\begin{itemize}
  \item \emph{Continuous reversibility:} between any two pure states there
        exists a continuous reversible transformation;
  \item \emph{Purification:} any mixed state of a system can be seen as the
        reduced state of a pure state of a larger system;
  \item \emph{Local tomography:} the state of a composite system is fully
        determined by the statistics of local measurements and their
        correlations.
\end{itemize}
Together these principles single out complex Hilbert space quantum theory
among generalized probabilistic theories. We therefore take as an
ontological fact that physical systems are described by density operators
on complex Hilbert spaces, with pure states corresponding to rays.

\subsection{A1: Systems and states}
\label{subsec:A1}

\textbf{(Skeleton placeholder)} This section will collect the precise
definition of physical systems, state spaces $\States$, and allowed
operations at the level of completely positive trace-preserving maps,
with composition rules for subsystems.

\subsection{A2: Composition and subsystems}
\label{subsec:A2}

\textbf{(Skeleton placeholder)} Here we specify how composite systems are
formed (tensor product structure), how subsystems are identified, and how
partial traces and reduced states are defined.

\subsection{A3: Measurements and statistics}
\label{subsec:A3}

\textbf{(Skeleton placeholder)} This section will state the assumptions
about POVMs, outcome probabilities and the statistical interpretation of
states, in a way compatible with A0 and with information-theoretic
quantities such as relative entropy and mutual information.

\subsection{A4: Resources and constraints}
\label{subsec:A4}

\textbf{(Skeleton placeholder)} This axiom will capture the idea that
physical processes are constrained by limited resources (energy, time,
computational complexity), preparing the way for A6 and A7.

\subsection{A5: Processes as channels and data-processing}
\label{subsec:A5}

Axiom A5 states that physically allowed processes act as information
channels that cannot increase distinguishability. At the level of quantum
states this is encoded in the data-processing inequality for relative
entropy and related monotone quantities.

\subsection{A6: Thermodynamic cost and Landauer principle}
\label{subsec:A6}

Axiom A6 asserts that erasure of information has an energetic cost.
More generally, there is a quantitative link between changes in
information-theoretic quantities (entropy, relative entropy) and
dissipated resources (heat), providing a microscopic origin of the
thermodynamic arrow of time.

\subsection{A7: Bounded computational complexity}
\label{subsec:A7}

Axiom A7 formalizes the idea that not all mathematically definable
states and transformations are physically realizable: there is a bound
on effective computational complexity. In the present v3.0 foundation we
treat this axiom at a qualitative level, as a bridge between IGD and
complexity-based constraints (such as those appearing in holography and
chaotic systems).

\section{Epistemic and geometric layer: focus, models and locality}
\label{sec:epistemic}

The second layer of the axiomatic structure concerns how an agent (or
effective theory) \emph{models} the underlying ontological physics. The
key concepts are focus, model manifold and informational locality.

\subsection{A8: Focus and model manifold}
\label{subsec:A8}

Axiom A8 introduces the notion of a \emph{focus}---a choice of coarse
description of the world in terms of a model manifold
$\Mtheta$ of macrostates (or effective parameters) $\theta$. The same
underlying ontic state can be represented by many different points in
different model manifolds, and the IGD equations must be covariant under
such changes of focus.

\subsection{A9: Informational locality and emergent geometry}
\label{subsec:A9}

Axiom A9 states that the global state space admits a factorization into
subsystems and that \emph{informational locality} is determined by the
pattern of correlations (e.g.\ mutual information) between them. This
induces an emergent graph structure and, in suitable limits, an emergent
spatial geometry on which interactions are quasi-local.

\subsection{A10: Geometric compatibility (K\"ahler structure)}
\label{subsec:A10}

Axiom A10 requires that the information metric $\gmat$, the symplectic
structure $\om$ and the complex structure $\Jc$ on the relevant state
manifolds form a compatible (K\"ahler-like) triple. This ensures that
unitary quantum evolution appears as an isometric Hamiltonian flow with
respect to the information metric, and allows us to write IGD in a
GENERIC-like form.

\section{Information--Geometric Dynamics}
\label{sec:igd_dynamics}

In this section we summarize how the axioms A0--A10 support the IGD
equation of motion for effective states $\theta(t)$ on a model manifold
$\Mtheta$:
\begin{equation}
  \dot{\theta}
  = \Jop(\theta)\,\grad \Efun(\theta)
  + \Rop(\theta)\,\grad \Sfun(\theta),
\end{equation}
where $\Jop$ is antisymmetric with respect to the metric $\gmat$,
implementing a Hamiltonian (reversible) flow, and $\Rop$ is symmetric
and positive semi-definite, implementing an entropy-increasing
(irreversible) flow. The precise conditions on $(\gmat,\om,\Jc,\Jop,\Rop)$
and their relation to the axioms will be elaborated in a dedicated
companion document.

\section{Preview: spacetime, gravity and complexity}
\label{sec:preview_spacetime}

Finally, we briefly indicate how the axioms laid out above are intended
to support the emergent picture of spacetime and gravity:

\begin{itemize}
  \item Theorems of ``area-law type'' (T1) relate informational locality
        and the structure of correlations to area-scaling of entanglement
        entropy in appropriate regimes.
  \item Thermodynamic and modular-geometry arguments (T2) aim to show
        that, under suitable locality and equilibrium assumptions, the
        emergent metric satisfies Einstein-like equations.
  \item Complexity-based considerations (T3) connect the growth of
        computational complexity along IGD trajectories to geometric
        quantities (such as volumes or actions) in the emergent
        spacetime description.
\end{itemize}

These results are \emph{not} taken as axioms; instead they are formulated
as target theorems in the spacetime programme built on top of the present
axiomatic foundation.

\end{document}
