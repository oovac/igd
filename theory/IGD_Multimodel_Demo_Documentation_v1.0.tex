\documentclass[11pt]{article}
\usepackage[margin=2.5cm]{geometry}
\usepackage[utf8]{inputenc}
\usepackage[T1]{fontenc}
\usepackage{hyperref}

\title{IGD Multimodel Simulator:\\
Usage and Mapping to Spacetime Theorems (v1.0)}
\author{Ordo Vacui and Collaborator}
\date{\today}

\begin{document}
\maketitle

\section{Overview}

The IGD multimodel simulator is a small Python codebase that implements
three complementary model sectors:

\begin{itemize}
  \item gapped spin chains (TFIM) as a realisation of the T1 area-law
        regime;
  \item critical 1+1D CFT behaviour and its relation to critical
        spin chains (TFIM at criticality);
  \item a small Majorana SYK$_4$--like model as a playground for
        chaos, scrambling and T3.
\end{itemize}

The main entry point is a script such as
\texttt{igd\_simulator\_v2.py} (or an equivalent front-end) providing
a command-line interface (CLI) to the different modes.

\section{TFIM mode (T1: gapped sector)}

In TFIM mode the simulator constructs the Hamiltonian of a
transverse-field Ising chain with parameters $(N,J,h)$ and computes:

\begin{itemize}
  \item the ground-state energy and gap $\Delta$;
  \item the information correlation length $\xi_{\mathrm{IGD}}$ and
        information velocity $v_{\mathrm{IGD}}$ in the gapped phase;
  \item static mutual information $I(d)$ between sites at distance
        $d$ in the ground state;
  \item block entropies $S_A(\ell)$ for contiguous intervals of
        length $\ell$ (T1 area-law diagnostic);
  \item dynamical entanglement growth $S_A(t)$ and mutual information
        light-cones $I(d,t)$ after a local or global quench.
\end{itemize}

These observables provide a direct numerical realisation of the
T1 spacetime theorem in one dimension. The corresponding IGD theory
is documented in \emph{IGD\_T1\_AreaLaw\_SpinChains\_v1.0.tex} and
\emph{IGD\_Lattice\_SpinChains\_v1.0.tex}.

\section{CFT mode (critical sector)}

In CFT mode the simulator evaluates analytical entanglement formulas
for a 1+1D conformal field theory with central charge $c$, including:

\begin{itemize}
  \item the ground-state entanglement entropy $S_A(\ell)$ of an
        interval of length $\ell$, with the characteristic
        logarithmic scaling $S_A(\ell) \sim \tfrac{c}{3}\log(\ell/a)$;
  \item quench entanglement growth $S_A(t)$ in simple CFT quench
        protocols.
\end{itemize}

An optional comparison mode juxtaposes these CFT predictions with
numerical results for a critical TFIM chain (e.g.\ $J=h$) of finite
size, allowing one to see explicitly how the lattice model approaches
the CFT scaling forms.

The IGD interpretation of this critical sector, and its role as a
bridge towards T2 (entanglement thermodynamics and Einstein-like
equations), is outlined in \emph{IGD\_CFT\_1p1\_v1.0.tex} and related
notes.

\section{SYK mode (T3: chaotic sector)}

In SYK mode the simulator provides two related chaotic models:

\begin{itemize}
  \item a random-Hermitian toy on $N_q$ qubits, useful as a simple
        test of entanglement growth;
  \item a structured Majorana SYK$_4$--like model built from $2N_f$
        Majorana operators on $N_f$ qubits, with Hamiltonian
        \(
          H = \sum_{i<j<k<l} J_{ijkl}\gamma_i\gamma_j\gamma_k\gamma_l.
        \)
\end{itemize}

For the SYK$_4$ toy the simulator computes:

\begin{itemize}
  \item half-system entanglement entropy $S_A(t)$ as a function of
        time, starting from a simple product state;
  \item an out-of-time-order correlator (OTOC) diagnostic
        $F(t)$ and the associated scrambling measure $C(t)=1-\Re F(t)$
        for simple local Pauli operators.
\end{itemize}

These outputs constitute a concrete, finite-size realisation of the
T3 phenomenology: rapid complexity growth and scrambling in a chaotic
IGD regime. The corresponding theory is discussed in
\emph{IGD\_Chaos\_Complexity\_T3\_v1.0.tex}.

\section{Mapping summary}

To summarise the mapping between simulator modes and spacetime
theorems:

\begin{center}
\begin{tabular}{lll}
\hline
Mode & Theorem / regime & Key observables \\
\hline
TFIM (gapped) & T1 (area law) &
  $S_A(\ell)$, $I(d)$, $S_A(t)$, $I(d,t)$ \\
CFT / critical TFIM & T1/T2 bridge (criticality) &
  $\log$-scaling $S_A(\ell)$, CFT quench $S_A(t)$ \\
SYK$_4$ (chaotic) & T3 (chaos / complexity) &
  $S_A(t)$, OTOC-based $C(t)$ \\
\hline
\end{tabular}
\end{center}

This documentation should be read together with the core dynamics
note \emph{IGD\_Core\_Dynamics\_and\_Theorems\_v3.1.tex} and the
model-specific theory files for a complete picture of the IGD
programme implemented in the code.
\end{document}
