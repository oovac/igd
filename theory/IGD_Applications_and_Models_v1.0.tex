\documentclass[11pt]{article}
\usepackage[margin=2.5cm]{geometry}
\usepackage[utf8]{inputenc}
\usepackage[T1]{fontenc}
\usepackage{amsmath,amssymb,amsfonts}
\usepackage{bm}
\usepackage{hyperref}

\title{Information--Geometric Dynamics:\\
Applications and Model Sectors (v1.0)}
\author{Ordo Vacui and Collaborator}
\date{\today}

\begin{document}
\maketitle

\begin{abstract}
This note provides a meta-level overview of the current model sectors
used to instantiate Information--Geometric Dynamics (IGD). It is meant
to sit between the axiomatic foundation and the detailed model-specific
manuscripts. We emphasise three complementary regimes:
(1) gapped lattice systems (T1, area laws),
(2) critical conformal regimes (bridge to T2, entanglement
thermodynamics),
(3) chaotic models of SYK type (T3, chaos and complexity).
Together they form an ``IGD phase diagram'' of applications.
\end{abstract}

\section{Position in the IGD stack}

The full IGD programme can be viewed as a layered stack:

\begin{itemize}
  \item \textbf{Axioms and core dynamics:} expressed in terms of
        axioms A0--A10 and the GENERIC IGD equation
        $\dot{\theta} = J(\theta)\nabla E(\theta) + R(\theta)\nabla S(\theta)$.
  \item \textbf{Spacetime theorems:} T1 (area laws),
        T2 (entanglement thermodynamics and Einstein-like equations),
        T3 (chaos/complexity and emergent geometry).
  \item \textbf{Applications and models:} concrete quantum systems in
        which the abstract structures can be tested and visualised.
\end{itemize}

This document focuses on the third layer: it explains which models we
use as representatives of different regimes, how they map to T1--T3,
and how they interrelate.

\section{Gapped lattice spin chains (T1 sector)}

The first and most controlled application sector consists of
one-dimensional gapped lattice models, in particular the transverse
field Ising model (TFIM) and its variants. The key features are:

\begin{itemize}
  \item a local Hamiltonian with a non-zero spectral gap $\Delta > 0$;
  \item exponentially decaying correlations with a finite correlation
        length $\xi_{\mathrm{IGD}}$;
  \item entanglement area laws: for the ground state, the entropy
        $S_A(\ell)$ of a block of length $\ell$ saturates to a
        constant for $\ell \gg \xi_{\mathrm{IGD}}$.
\end{itemize}

In the IGD language this sector realises the spacetime theorem T1:
informational locality (A9) and the K\"ahler-compatible dynamics
(A10) together yield area-law behaviour for entanglement and mutual
information. The TFIM implementation provides:

\begin{itemize}
  \item analytic control via fermionisation in simple regimes;
  \item exact diagonalisation for small systems (the current simulator);
  \item a clean numerical demonstration of static and dynamical area
        laws (mutual-information light-cones, entanglement saturation).
\end{itemize}

The detailed analysis of this sector is carried out in the spin-chain
workplan and related drafts (e.g.\ the TFIM static and dynamical
preprints in the \texttt{paper/} directory).

\section{Critical CFT-like regimes (bridge to T2)}

The second sector consists of critical models with emergent conformal
symmetry, such as the TFIM at its quantum critical point and
1+1-dimensional conformal field theories (CFTs) with central charge
$c$. Here the spectral gap closes and the entanglement entropy exhibits
characteristic logarithmic scaling,
\begin{equation}
  S_A(\ell) \sim \frac{c}{3}\log(\ell/a) + \text{const}.
\end{equation}

This sector plays a dual role:

\begin{itemize}
  \item it marks the boundary between strictly gapped T1 regimes and
        more subtle gapless physics;
  \item it provides the natural playground for T2, where
        entanglement thermodynamics and modular Hamiltonians are
        expected to give rise to Einstein-like equations for emergent
        metrics.
\end{itemize}

On the lattice side, the critical TFIM provides a concrete
approximation to a 1+1D CFT. On the continuum side, explicit CFT
formulas for $S_A(\ell)$ and quench entanglement $S_A(t)$ allow us to
compare lattice numerics with analytic predictions.

A dedicated CFT/critical manuscript (e.g.\
\emph{IGD\_CFT\_1p1\_v1.0.tex}) is meant to accompany this overview,
detailing how IGD structures appear in conformal settings and how they
relate to T2.

\section{Chaotic SYK-type models (T3 sector)}

The third sector targets strongly chaotic dynamics and fast scrambling.
The guiding example is the Sachdev--Ye--Kitaev (SYK) family of models,
in which random all-to-all interactions between fermions produce
maximally chaotic behaviour and, in suitable limits, holographic
duality to nearly-AdS$_2$ gravity.

In the present codebase we use small SYK-like toy models as a T3
playground:

\begin{itemize}
  \item a random-Hermitian toy Hamiltonian on a few qubits, used to
        illustrate generic entanglement growth in chaotic systems;
  \item (in planned extensions) small Majorana SYK$_4$-like models,
        where a Hamiltonian of the form
        $H = \sum_{i<j<k<l} J_{ijkl}\gamma_i\gamma_j\gamma_k\gamma_l$
        can be implemented on a handful of qubits and studied via
        exact diagonalisation.
\end{itemize}

The main IGD diagnostics in this sector are:

\begin{itemize}
  \item half-system entanglement entropy $S_A(t)$ as a function of
        time, starting from simple product states;
  \item out-of-time-order correlator (OTOC) diagnostics $F(t)$ and
        scrambling measures $C(t) = 1-\Re F(t)$.
\end{itemize}

These quantities act as operational complexity proxies for the T3
spacetime theorem, which links complexity growth in IGD to emergent
geometric quantities. A more detailed chaos-focused manuscript (e.g.\
\emph{IGD\_SYK\_Chaos\_v1.0.tex}) is intended to expand on this sector.

\section{An IGD ``phase diagram'' of regimes}

It is conceptually useful to organise these sectors into an abstract
``phase diagram'' of IGD applications. Two natural axes are:

\begin{itemize}
  \item \textbf{spectral gap / correlation length:} from gapped,
        short-range-correlated phases to gapless, critical or even
        long-range-correlated regimes;
  \item \textbf{degree of chaos:} from integrable or nearly-integrable
        models to strongly chaotic, fast-scrambling systems.
\end{itemize}

In such a diagram:

\begin{itemize}
  \item gapped TFIM-like chains sit in the ``gapped / weakly chaotic''
        corner: they are local, admit area laws, and serve as clean
        T1 testbeds;
  \item critical TFIM and CFTs sit along a ``gapless but structured''
        line: they exhibit long-range correlations controlled by
        symmetry and scaling, and bridge towards T2 and entanglement
        thermodynamics;
  \item SYK-type models inhabit the ``strongly chaotic'' region:
        locality is absent or emergent, scrambling is fast, and the
        dominant IGD phenomenon is complexity growth (T3).
\end{itemize}

The IGD stack is designed so that all three corners are described
within a single information--geometric language: the same axioms,
the same IGD equation, and different coarse-graining regimes.

\section{Roadmap and next steps}

From the perspective of this applications layer, the immediate next
steps are:

\begin{itemize}
  \item \textbf{T1 paper (gapped regime):} consolidate the TFIM
        results---static mutual information, block entropies,
        dynamical light-cones---into a coherent manuscript that
        demonstrates T1 in full detail.
  \item \textbf{Critical / CFT bridge:} complete the CFT/critical note
        and strengthen the mapping between critical TFIM numerics and
        CFT entanglement formulas, preparing the ground for T2-style
        entanglement thermodynamics.
  \item \textbf{Chaos / T3 upgrade:} refine the SYK sector by
        implementing a small but genuine Majorana SYK$_4$ model with
        entanglement and OTOC diagnostics, bringing the toy chaos
        studies closer to the T3 spacetime theorem.
  \item \textbf{Unified narrative:} eventually combine these sectors
        into a single ``Applications and Models'' chapter of a
        larger IGD monograph, using this v1.0 note as the high-level
        table of contents for the model landscape.
\end{itemize}

In this way the IGD applications layer becomes a structured map:
from gapped to critical to chaotic regimes, all phrased in the same
information--geometric language and supported by explicit numerical
simulations.
\end{document}
