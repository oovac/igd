\documentclass[11pt]{article}
\usepackage[margin=2.5cm]{geometry}
\usepackage[utf8]{inputenc}
\usepackage[T1]{fontenc}
\usepackage[english]{babel}

% File: IGD_Notation_v1.tex
% Shared notation and symbol conventions for IGD papers.
%
% This file is intended to be \input in the preamble of each IGD LaTeX document.

\usepackage{amsmath,amssymb,amsfonts,amsthm}
\usepackage{bm}

% --- Sets, spaces, manifolds ---

% Hilbert space and state space
\newcommand{\Hilb}{\mathcal{H}}           % underlying Hilbert space
\newcommand{\States}{\mathcal{S}}         % space of physical states (density operators)
\newcommand{\Pure}{\mathcal{P}}           % manifold of pure states

% Parameter / model manifolds
\newcommand{\Mtheta}{\mathcal{M}_\theta}  % model (focus) manifold
\newcommand{\Mphys}{\mathcal{M}}          % generic state manifold

% Subsystems
\newcommand{\HA}{\Hilb_A}
\newcommand{\HB}{\Hilb_B}

% --- Geometry: metric, symplectic form, complex structure ---

\newcommand{\gij}{g_{ij}}                 % components of the information metric
\newcommand{\gmat}{g}                     % metric tensor
\newcommand{\om}{\omega}                  % symplectic 2-form
\newcommand{\Jc}{\mathcal{J}}             % (almost) complex structure on pure states
\newcommand{\Jop}{J}                      % antisymmetric operator implementing Hamiltonian flow
\newcommand{\Rop}{R}                      % symmetric operator implementing dissipative flow

% --- Energetic and entropic functionals ---

\newcommand{\Efun}{E}                     % energy functional on \Mtheta
\newcommand{\Sfun}{S}                     % entropy functional on \Mtheta

% --- Gradients and flows ---

\newcommand{\grad}{\nabla}                % gradient with respect to g
\newcommand{\dotth}{\dot{\theta}}        % time derivative of parameters
\newcommand{\ddt}{\frac{d}{dt}}          % total time derivative

% --- Information-theoretic quantities ---

\newcommand{\MI}{I}                       % mutual information
\newcommand{\SX}{S}                       % entropy (generic)
\newcommand{\KB}{k_{\mathrm{B}}}          % Boltzmann constant

\newcommand{\xiIGD}{\xi}                  % correlation / information length scale
\newcommand{\vIGD}{v_{\mathrm{IGD}}}      % information velocity (Lieb--Robinson-like)

% --- Misc ---

\newcommand{\Tr}{\mathrm{Tr}}
\newcommand{\id}{\mathbbm{1}}


\title{Information--Geometric Dynamics v3.0:\\
Foundations, Dynamics and Applications}
\author{Ordo Vacui and Collaborator}
\date{\today}

\begin{document}
\maketitle

\begin{abstract}
This master document provides a high-level map of the current
Information--Geometric Dynamics (IGD) framework in version 3.0.
Rather than duplicating material, it points to three core texts:
the axiomatic foundation, the core dynamics and spacetime
theorems, and a collection of model applications (TFIM, CFT and
SYK-like toy models). It is intended as an entry point for new
readers and collaborators.
\end{abstract}

\section{Structure of the IGD v3.0 theory stack}

The IGD v3.0 theory is organized into three main layers:

\begin{enumerate}
  \item \textbf{Foundations:} axioms A0--A10, separating ontological
        statements about quantum information from epistemic/geometric
        assumptions about focus manifolds and locality.
  \item \textbf{Core dynamics and spacetime theorems:} the IGD
        equation of motion in GENERIC-like form and the statements
        of spacetime-related theorems T1--T3.
  \item \textbf{Applications and models:} concrete instantiations of
        IGD in lattice models, conformal field theories and chaotic
        systems.
\end{enumerate}

Each layer is currently implemented as a separate \LaTeX{} document in
the \texttt{theory/} directory of the repository.

\section{Foundations}
\label{sec:foundations}

The axiomatic basis of IGD is presented in:
\begin{itemize}
  \item \texttt{IGD\_Axiomatic\_Foundation\_v3.0.tex}
\end{itemize}

This text introduces:
\begin{itemize}
  \item A0: quantum structure (complex Hilbert space, purification,
        local tomography);
  \item A1--A4: systems, composition, measurements and resource
        constraints (to be further elaborated);
  \item A5--A7: processes as channels, Landauer/thermodynamic cost
        and bounded computational complexity;
  \item A8--A10: focus manifolds, informational locality and
        K\"ahler-compatible information geometry.
\end{itemize}

It also summarizes how these axioms support an IGD equation of motion
for effective states on focus manifolds, and briefly previews the
spacetime programme.

\section{Core dynamics and spacetime theorems}
\label{sec:core}

The core dynamical structure and the target spacetime theorems are
described in:
\begin{itemize}
  \item \texttt{IGD\_Core\_Dynamics\_and\_Theorems\_v3.0.tex}
\end{itemize}

This document:
\begin{itemize}
  \item formulates the IGD equation
        \begin{equation}
          \dot{\theta}
          = \Jop(\theta)\,\grad \Efun(\theta)
          + \Rop(\theta)\,\grad \Sfun(\theta),
        \end{equation}
        with structural constraints on $\Jop$ and $\Rop$ (antisymmetry,
        symmetry, positivity, degeneracy conditions);
  \item explains how entropy production and the arrow of time arise
        from the irreversible part of the flow;
  \item states the three spacetime theorems (in target form):
        \begin{itemize}
          \item T1: informational area laws and holographic bounds;
          \item T2: Einstein-like equations from entanglement
                thermodynamics;
          \item T3: complexity growth and emergent geometry.
        \end{itemize}
\end{itemize}

At this stage, T1--T3 are formulated as precise goals rather than
fully proven results; the document clarifies the assumptions and
regimes under which they are expected to hold.

\section{Applications and models}
\label{sec:applications}

Concrete model realizations of IGD are collected in:
\begin{itemize}
  \item \texttt{IGD\_Applications\_and\_Models\_v1.0.tex}
\end{itemize}

That text organizes three main regimes:

\begin{enumerate}
  \item \textbf{Gapped lattice models (TFIM):} a one-dimensional
        transverse-field Ising chain in its gapped phase provides
        a clean example of informational locality (A9), finite
        information length $\xiIGD$, and an information velocity
        $\vIGD$ controlling light-cone-like spreading of mutual
        information. Both static (ground-state mutual information)
        and dynamic (entanglement growth and light-cones after a
        quench) aspects are discussed.
  \item \textbf{1+1D CFTs:} conformal field theories in two
        dimensions realize scale-invariant IGD regimes, with
        universal entanglement entropies for intervals and
        light-cone-like entanglement growth after global quenches.
        They connect lattice IGD to continuum spacetime descriptions.
  \item \textbf{SYK-like chaotic toys:} random all-to-all Hamiltonians
        on few-qubit Hilbert spaces illustrate fast scrambling and
        near-maximal entanglement, serving as a playground for the
        complexity/geometry conjecture T3.
\end{enumerate}

Together, these models span a qualitative phase diagram of IGD
behaviour: gapped local, critical conformal and strongly chaotic.

\section{Reading guide and future integration}
\label{sec:reading}

For a new reader, a natural path through the IGD stack is:
\begin{enumerate}
  \item skim the present master document to understand the overall
        structure;
  \item read the axiomatic foundation (v3.0) to internalize A0--A10;
  \item study the core dynamics and spacetime theorems (v3.0) to see
        how IGD flows and what the target geometric results are;
  \item explore the applications and models (v1.0) alongside the
        multimodel demo code to build intuition.
\end{enumerate}

In future revisions, the separate \LaTeX{} texts may be refactored
into chapters of a single book-style document. For now they are kept
modular to facilitate parallel development and experimentation.

\end{document}
