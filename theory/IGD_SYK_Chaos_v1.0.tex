\documentclass[11pt]{article}
\usepackage[margin=2.5cm]{geometry}
\usepackage[utf8]{inputenc}
\usepackage[T1]{fontenc}
\usepackage[english]{babel}

% File: IGD_Notation_v1.tex
% Shared notation and symbol conventions for IGD papers.
%
% This file is intended to be \input in the preamble of each IGD LaTeX document.

\usepackage{amsmath,amssymb,amsfonts,amsthm}
\usepackage{bm}

% --- Sets, spaces, manifolds ---

% Hilbert space and state space
\newcommand{\Hilb}{\mathcal{H}}           % underlying Hilbert space
\newcommand{\States}{\mathcal{S}}         % space of physical states (density operators)
\newcommand{\Pure}{\mathcal{P}}           % manifold of pure states

% Parameter / model manifolds
\newcommand{\Mtheta}{\mathcal{M}_\theta}  % model (focus) manifold
\newcommand{\Mphys}{\mathcal{M}}          % generic state manifold

% Subsystems
\newcommand{\HA}{\Hilb_A}
\newcommand{\HB}{\Hilb_B}

% --- Geometry: metric, symplectic form, complex structure ---

\newcommand{\gij}{g_{ij}}                 % components of the information metric
\newcommand{\gmat}{g}                     % metric tensor
\newcommand{\om}{\omega}                  % symplectic 2-form
\newcommand{\Jc}{\mathcal{J}}             % (almost) complex structure on pure states
\newcommand{\Jop}{J}                      % antisymmetric operator implementing Hamiltonian flow
\newcommand{\Rop}{R}                      % symmetric operator implementing dissipative flow

% --- Energetic and entropic functionals ---

\newcommand{\Efun}{E}                     % energy functional on \Mtheta
\newcommand{\Sfun}{S}                     % entropy functional on \Mtheta

% --- Gradients and flows ---

\newcommand{\grad}{\nabla}                % gradient with respect to g
\newcommand{\dotth}{\dot{\theta}}        % time derivative of parameters
\newcommand{\ddt}{\frac{d}{dt}}          % total time derivative

% --- Information-theoretic quantities ---

\newcommand{\MI}{I}                       % mutual information
\newcommand{\SX}{S}                       % entropy (generic)
\newcommand{\KB}{k_{\mathrm{B}}}          % Boltzmann constant

\newcommand{\xiIGD}{\xi}                  % correlation / information length scale
\newcommand{\vIGD}{v_{\mathrm{IGD}}}      % information velocity (Lieb--Robinson-like)

% --- Misc ---

\newcommand{\Tr}{\mathrm{Tr}}
\newcommand{\id}{\mathbbm{1}}


\title{IGD Applications: Chaotic Dynamics and SYK-like Models (v1.0)}
\author{Ordo Vacui and Collaborator}
\date{\today}

\begin{document}
\maketitle

\begin{abstract}
We outline how Information--Geometric Dynamics (IGD) can be applied
to strongly chaotic quantum systems, using SYK-like models as a
guiding example. In such systems entanglement entropies of subsystems
grow rapidly towards near-maximal values, and operator complexity
increases approximately linearly over long times. We interpret
these phenomena through the IGD lens, emphasizing the role of
A7 (bounded computational complexity) and the target theorem T3
(complexity/geometry correspondence).
\end{abstract}

\tableofcontents

\section{Motivation}
\label{sec:syk_intro}

While gapped spin chains and CFTs illustrate IGD in relatively
controlled regimes (locality with finite or scale-invariant correlation
lengths), strongly chaotic models push IGD into a different corner of
the landscape. SYK-type models, with all-to-all random interactions
and fast scrambling dynamics, are natural candidates for exploring
the interplay between:

\begin{itemize}
  \item rapid entanglement growth and saturation;
  \item exponential operator growth and linear complexity growth;
  \item emergent geometric descriptions in terms of nearly-AdS$_2$
        or related spaces.
\end{itemize}

In this v1.0 note we do not attempt a full SYK construction; instead
we sketch the IGD interpretation and connect to a simple toy model
implemented in the multimodel demo repository.

\section{Toy model: random all-to-all Hamiltonian}
\label{sec:syk_toy_model}

As a minimal chaotic playground we can take a random all-to-all
Hermitian Hamiltonian acting on $N_q$ qubits:
\begin{equation}
  H_{\text{toy}} = H_{\text{ran}},
\end{equation}
where $H_{\text{ran}}$ is drawn from a suitable ensemble (e.g.\ a
Gaussian unitary ensemble on the $2^{N_q}$-dimensional Hilbert space).
Starting from a simple product state
\begin{equation}
  |\psi(0)\rangle = |\uparrow \uparrow \cdots \uparrow\rangle_z,
\end{equation}
we consider the unitary evolution
\begin{equation}
  |\psi(t)\rangle = e^{-i H_{\text{toy}} t} |\psi(0)\rangle.
\end{equation}

For moderate $N_q$ (e.g.\ $N_q=6$), one can diagonalize $H_{\text{toy}}$
exactly and compute the entanglement entropy $S_A(t)$ of half the
qubits as a function of time. Numerically, one typically observes:

\begin{itemize}
  \item a rapid initial growth of $S_A(t)$ from zero towards a value
        close to the Page entropy of a random pure state;
  \item fluctuations around this near-maximal entropy at late times.
\end{itemize}

This behaviour contrasts with the light-cone-limited growth in local
models: here, entanglement spreads essentially without spatial
constraints, reflecting the all-to-all connectivity.

\section{IGD interpretation: complexity and T3}
\label{sec:syk_igd}

In IGD terms, the toy model and more realistic SYK models highlight
the role of A7 (bounded computational complexity) and the T3
complexity/geometry conjecture:

\begin{itemize}
  \item The focus manifold $\Mtheta$ can be taken as a space of pure
        states or unitaries equipped with a \emph{complexity metric}
        (e.g.\ Nielsen geometry), in addition to an information metric.
  \item The IGD flow $\dot{\theta} = \Jop \grad \Efun + \Rop \grad \Sfun$
        then has both an information-geometric and a complexity-geometric
        interpretation: trajectories explore increasingly complex regions
        of $\Mtheta$ as $t$ grows.
  \item In chaotic regimes, the complexity $\mathcal{C}(t)$ of
        $|\psi(t)\rangle$ (or of the corresponding unitary) is
        expected to grow approximately linearly over long times,
        before saturating due to the finite Hilbert space size.
\end{itemize}

T3 posits that this complexity growth can be related to geometric
quantities in an emergent spacetime description, such as volumes of
maximal slices or on-shell actions in a dual gravitational theory:
\begin{equation}
  \mathcal{C}(t) \propto V(t)
  \quad\text{or}\quad
  \mathcal{C}(t) \propto \mathcal{A}(t).
\end{equation}
From the IGD perspective, this is not taken as an independent axiom,
but as a structural pattern to be derived in suitable limits.

\section{Connection to the multimodel demo}
\label{sec:syk_demo}

The multimodel demo accompanying this theory includes a small
SYK-like toy implementation: a random Hermitian Hamiltonian on
$N_q$ qubits, exact diagonalization, and computation of the
half-system entropy $S_A(t)$. While this toy does not capture the
full richness of SYK (e.g.\ Majorana fermions, specific interaction
structure), it already illustrates:

\begin{itemize}
  \item fast scrambling of entanglement compared to local models;
  \item the absence of a simple light-cone structure in real space;
  \item the relevance of global complexity-based notions of distance
        on state space.
\end{itemize}

This makes the toy model a useful IGD benchmark for the chaotic
corner of the theory, complementary to the TFIM and CFT regimes.

\section{Outlook}
\label{sec:syk_outlook}

A full IGD treatment of SYK proper would involve:

\begin{itemize}
  \item specifying appropriate focus manifolds (e.g.\ low-energy
        effective descriptions, Schwarzian sectors);
  \item constructing explicit information and complexity metrics;
  \item analyzing how the IGD flow reproduces known features such as
        maximal chaos and nearly-AdS$_2$ emergent geometries.
\end{itemize}

The present v1.0 note is intended as a stepping stone: it anchors the
chaotic, complexity-dominated regime in the IGD framework and clarifies
the conceptual link to the T3 complexity/geometry conjecture.

\end{document}
