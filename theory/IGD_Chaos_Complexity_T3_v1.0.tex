\documentclass[11pt]{article}
\usepackage[margin=2.5cm]{geometry}
\usepackage[utf8]{inputenc}
\usepackage[T1]{fontenc}
\usepackage{amsmath,amssymb,amsfonts}
\usepackage{bm}
\usepackage{hyperref}

\title{IGD Spacetime Theorem T3:\\
Chaos, Complexity and Information Geometry\\(v1.0)}
\author{Ordo Vacui and Collaborator}
\date{\today}

\begin{document}
\maketitle

\begin{abstract}
We formulate the IGD spacetime theorem T3 in an explicit, operational
way and connect it to concrete diagnostics in small chaotic quantum
systems. The theorem links the growth of computational complexity
and scrambling---as probed by entanglement and out-of-time-order
correlators (OTOCs)---to emergent geometric quantities in the
information--geometric description of dynamics. As a minimal
playground we use a small Majorana SYK$_4$--like model implemented on
a few qubits, for which we can compute entanglement and OTOC dynamics
exactly.
\end{abstract}

\section{Setting and motivation}

Within the Information--Geometric Dynamics (IGD) framework the
microscopic state space is the quantum state space $\mathcal{S}$ of
density operators on a Hilbert space $\mathcal{H}$, equipped with an
information metric (e.g.\ the Bures metric) and a compatible
symplectic structure. Axioms A0--A10 guarantee a well-defined quantum
structure (Hilbert space, purification, local tomography), locality on
an information graph, and a K\"ahler-compatible geometry for the
reversible dynamics.

The spacetime theorems T1 and T2 address, respectively, entanglement
area laws and the emergence of Einstein-like field equations from
entanglement thermodynamics in appropriate regimes. The third
spacetime theorem T3 is meant to capture the link between
\emph{computational complexity} of quantum states and the emergent
geometry of spacetime.

In practice, for finite systems, ``complexity'' must be probed through
operational diagnostics. Two natural proxies are:

\begin{itemize}
  \item entanglement entropies of subsystems $S_A(t)$ under unitary
        or IGD evolution;
  \item out-of-time-order correlators (OTOCs)
        $F(t) = \langle W^\dagger(t) V^\dagger W(t) V \rangle$ and
        their associated scrambling measures, e.g.\
        $C(t) = 1 - \Re F(t)$.
\end{itemize}

\section{Informal statement of T3}

We give here an informal formulation of T3 adapted to finite
chaotic systems.

\medskip\noindent
\textbf{T3 (Chaos, complexity and emergent geometry, informal).}
\emph{
Consider a chaotic quantum system satisfying the IGD axioms A0--A10
and the locality assumptions entering T1 and T2. Let
$\{\rho(t)\}_{t \ge 0}$ be the IGD evolution generated by a local
Hamiltonian plus possible weak dissipative terms, and let
$\mathcal{M}_\theta$ be a focus manifold capturing the physically
accessible states (e.g.\ low-energy or few-body states).

Then, in regimes of fast scrambling, there exists an effective
geometric description in which:
\begin{enumerate}
  \item the growth of a suitable complexity functional
        $\mathcal{C}[\rho(t)]$ along the IGD flow is controlled by
        geometric quantities (e.g.\ volumes or actions) in an
        emergent spacetime domain;
  \item operational diagnostics such as entanglement growth
        $S_A(t)$ and OTOC-based scrambling measures $C(t)$ track
        the growth of $\mathcal{C}[\rho(t)]$ up to model-dependent
        but bounded distortions;
  \item saturation of complexity and scrambling corresponds to
        saturation of geometric quantities (e.g.\ maximal slices or
        extremal actions) in the emergent description.
\end{enumerate}
}

\medskip

In other words, T3 asserts that in sufficiently chaotic regimes, the
information--geometric length of trajectories in $\mathcal{M}_\theta$
and the growth of operational complexity proxies are not arbitrary but
are encoded in the geometry of an emergent spacetime associated with
the state and its IGD evolution.

\section{Majorana SYK$_4$ toy model as a T3 playground}

As a minimal testbed we consider a small Majorana SYK$_4$--like
model implemented on $N_f$ fermionic modes (encoded on $N_f$ qubits).
We introduce $2N_f$ Majorana operators $\gamma_i$ via a Jordan--Wigner
construction and define the Hamiltonian
\begin{equation}
  H_{\mathrm{SYK4}} = \sum_{i<j<k<l} J_{ijkl}
    \,\gamma_i \gamma_j \gamma_k \gamma_l,
\end{equation}
with real random couplings $J_{ijkl}$ drawn from a Gaussian
distribution (with appropriate scaling) and fixed once and for all.

The IGD simulator implements this model for small $N_f$ (e.g.\ $N_f=4$)
and allows one to compute:

\begin{itemize}
  \item the half-system entanglement entropy $S_A(t)$ under unitary
        evolution generated by $H_{\mathrm{SYK4}}$ starting from a
        simple product state in the $Z$ basis;
  \item an OTOC-based scrambling diagnostic
        \begin{equation}
          F(t) = \frac{1}{d}\,\mathrm{Tr}[W(t) V W(t) V],\qquad
          C(t) = 1-\Re F(t),
        \end{equation}
        where $d$ is the Hilbert space dimension, $V$ and $W$ are
        simple local Pauli operators (e.g.\ $Z$ on the first and last
        qubit), and $W(t) = e^{+iHt} W e^{-iHt}$ is the Heisenberg
        evolution.
\end{itemize}

For sufficiently generic couplings one observes:

\begin{itemize}
  \item rapid growth of $S_A(t)$ from near-zero values to a plateau
        close to the Page value for a half-system, indicating fast
        scrambling of local information;
  \item an OTOC diagnostic $C(t)$ that grows from $0$ towards
        $\mathcal{O}(1)$ on a timescale set by the inverse
        interaction strength, signalling the decay of simple operator
        commutativity and the spreading of operators across the
        system.
\end{itemize}

These behaviours are the finite-size analogues of the more familiar
large-$N$ SYK phenomenology and provide a concrete, exactly solvable
playground for T3.

\section{Complexity proxies and IGD geometry}

In the full IGD picture, one would like to identify a genuine
complexity functional $\mathcal{C}[\rho]$ associated with the length
of geodesics in an appropriate information--geometric metric on the
space of states or unitaries. For small systems it is often more
practical to work with proxies:

\begin{itemize}
  \item \textbf{Entanglement-based proxies:} the half-system entropy
        $S_A(t)$ is a crude but robust measure of how ``spread out''
        information has become. In the SYK$_4$ toy it grows from
        nearly zero to a plateau near its maximum value, mirroring the
        expected growth of circuit complexity in chaotic dynamics.
  \item \textbf{OTOC-based proxies:} the scrambling diagnostic $C(t)$
        captures the growth of operator complexity in the Heisenberg
        picture. The early-time behaviour of $C(t)$ is sensitive to
        Lyapunov-like exponents in appropriate limits, while the
        saturation regime reflects the effective mixing of the
        operator algebra.
\end{itemize}

From the IGD standpoint, both $S_A(t)$ and $C(t)$ can be interpreted
as projections of the full information--geometric motion onto specific
low-dimensional observables. T3 then asserts that there exists an
emergent geometric description in which the increase of these
proxies corresponds to the growth of volumes or actions of certain
spacelike or timelike regions.

\section{Outlook and role of the simulator}

The current multimodel IGD simulator implements:

\begin{itemize}
  \item gapped spin chains (TFIM) as a T1 testbed;
  \item critical 1+1D CFT behaviour via the critical TFIM and CFT
        entanglement formulas;
  \item a small Majorana SYK$_4$ toy with entanglement and OTOC
        diagnostics as a T3 playground.
\end{itemize}

For the SYK$_4$ sector, the simulator provides exact spectra, exact
time evolution, and numerically exact $S_A(t)$ and $C(t)$ curves for
small $N_f$. These curves serve as concrete instantiations of the T3
phenomenology: rapid complexity growth in a chaotic IGD regime
tracked by entanglement and OTOCs.

Future work will refine the geometric side of T3, for example by:

\begin{itemize}
  \item constructing explicit information metrics on the manifold of
        unitaries generated by $H_{\mathrm{SYK4}}$ and relating their
        geodesic distances to $S_A(t)$ and $C(t)$;
  \item exploring coarse-grained emergent geometries in which the SYK
        dynamics corresponds to geodesic or gradient flows;
  \item connecting the finite-size toy model to large-$N$ SYK
        results and to holographic proposals such as
        complexity--volume and complexity--action.
\end{itemize}

In the present v1.0 note, our goal is more modest: to anchor T3 in a
precise yet operationally accessible setting, and to tie it directly
to the outputs of the IGD multimodel simulator.
\end{document}
