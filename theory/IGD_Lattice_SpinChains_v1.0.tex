\documentclass[11pt]{article}
\usepackage[margin=2.5cm]{geometry}
\usepackage[utf8]{inputenc}
\usepackage[T1]{fontenc}
\usepackage[english]{babel}

% File: IGD_Notation_v1.tex
% Shared notation and symbol conventions for IGD papers.
%
% This file is intended to be \input in the preamble of each IGD LaTeX document.

\usepackage{amsmath,amssymb,amsfonts,amsthm}
\usepackage{bm}

% --- Sets, spaces, manifolds ---

% Hilbert space and state space
\newcommand{\Hilb}{\mathcal{H}}           % underlying Hilbert space
\newcommand{\States}{\mathcal{S}}         % space of physical states (density operators)
\newcommand{\Pure}{\mathcal{P}}           % manifold of pure states

% Parameter / model manifolds
\newcommand{\Mtheta}{\mathcal{M}_\theta}  % model (focus) manifold
\newcommand{\Mphys}{\mathcal{M}}          % generic state manifold

% Subsystems
\newcommand{\HA}{\Hilb_A}
\newcommand{\HB}{\Hilb_B}

% --- Geometry: metric, symplectic form, complex structure ---

\newcommand{\gij}{g_{ij}}                 % components of the information metric
\newcommand{\gmat}{g}                     % metric tensor
\newcommand{\om}{\omega}                  % symplectic 2-form
\newcommand{\Jc}{\mathcal{J}}             % (almost) complex structure on pure states
\newcommand{\Jop}{J}                      % antisymmetric operator implementing Hamiltonian flow
\newcommand{\Rop}{R}                      % symmetric operator implementing dissipative flow

% --- Energetic and entropic functionals ---

\newcommand{\Efun}{E}                     % energy functional on \Mtheta
\newcommand{\Sfun}{S}                     % entropy functional on \Mtheta

% --- Gradients and flows ---

\newcommand{\grad}{\nabla}                % gradient with respect to g
\newcommand{\dotth}{\dot{\theta}}        % time derivative of parameters
\newcommand{\ddt}{\frac{d}{dt}}          % total time derivative

% --- Information-theoretic quantities ---

\newcommand{\MI}{I}                       % mutual information
\newcommand{\SX}{S}                       % entropy (generic)
\newcommand{\KB}{k_{\mathrm{B}}}          % Boltzmann constant

\newcommand{\xiIGD}{\xi}                  % correlation / information length scale
\newcommand{\vIGD}{v_{\mathrm{IGD}}}      % information velocity (Lieb--Robinson-like)

% --- Misc ---

\newcommand{\Tr}{\mathrm{Tr}}
\newcommand{\id}{\mathbbm{1}}


\title{IGD Applications: Gapped Spin Chains and the TFIM (v1.0)}
\author{Ordo Vacui and Collaborator}
\date{\today}

\begin{document}
\maketitle

\begin{abstract}
We present the transverse-field Ising model (TFIM) as a concrete lattice
realization of Information--Geometric Dynamics (IGD). In a gapped
ferromagnetic phase the model exhibits informational locality (A9),
a finite information length scale $\xiIGD$, and a Lieb--Robinson-like
information velocity $\vIGD$. We link exact analytic properties of the
TFIM (gap, correlation length, dispersion) to the IGD structures and
summarize numerical demonstrations of ground-state mutual information
and time-dependent information light-cones.
\end{abstract}

\section{Model and parameter regime}
\label{sec:tfim_model}

We consider the one-dimensional TFIM with open boundary conditions
\begin{equation}
  H = - J \sum_{i=1}^{N-1} \sigma^z_i \sigma^z_{i+1}
      - h \sum_{i=1}^{N} \sigma^x_i,
\end{equation}
with $J>0$ and transverse field $h>0$. Throughout we focus on the
gapped ferromagnetic phase $J>h$, and in particular on the working
point
\begin{equation}
  J = 1, \qquad h = 0.5.
\end{equation}
Via a Jordan--Wigner transformation and Bogoliubov rotation the model
maps to free fermions with dispersion relation
\begin{equation}
  \epsilon_k = 2 \sqrt{J^2 + h^2 - 2Jh \cos k},
\end{equation}
yielding a spectral gap
\begin{equation}
  \Delta = \min_k \epsilon_k = 2|J-h|
\end{equation}
and a correlation length
\begin{equation}
  \xiIGD^{-1} = \ln(J/h).
\end{equation}
For $J=1$, $h=0.5$ one finds
\begin{equation}
  \Delta = 1,
  \qquad
  \xiIGD = \frac{1}{\ln 2} \approx 1.44.
\end{equation}

The maximum group velocity of quasiparticles reads
\begin{equation}
  v_{\max} = 2 \min(J,h) =
  \begin{cases}
    2h, & J>h,\\
    2J, & h>J,
  \end{cases}
\end{equation}
which at the working point $J>h$ reduces to $v_{\max}=2h=1$.
We identify this with the lattice information velocity
\begin{equation}
  \vIGD = v_{\max} = 1.
\end{equation}

\section{IGD perspective: focus manifold and geometry}
\label{sec:tfim_focus}

From the IGD point of view, the microscopic state space $\States$ of
the TFIM consists of density operators on the spin chain. A natural
focus manifold $\Mtheta$ for low-energy physics is the manifold of
Gaussian fermionic states obtained after Jordan--Wigner and Bogoliubov
transformations. In this Gaussian focus:

\begin{itemize}
  \item The state is fully characterized by a covariance matrix
        $\Gamma$, whose entries serve as coordinates $\theta$.
  \item The Bures (or Fubini--Study) metric induces a Riemannian metric
        $\gmat$ on the manifold of pure Gaussian states; the fermionic
        canonical structure provides $\om$ and $\Jc$, leading to a
        K\"ahler-like structure as required by A10.
  \item The reversible IGD flow $\Jop \grad \Efun$ corresponds to
        unitary evolution under the quadratic Hamiltonian in the
        Gaussian variables, while the irreversible part $\Rop \grad
        \Sfun$ can be associated with Gaussian-preserving dissipative
        channels.
\end{itemize}

Thus the TFIM in the gapped phase provides a clean example where the
IGD geometry can be made explicit.

\section{Static informational locality: mutual information and area law}
\label{sec:tfim_static}

Let $|\psi_0\rangle$ denote the ground state of $H$ in the gapped
phase. For sites $i$ and $j$, we define reduced density matrices
$\rho_i$, $\rho_j$, $\rho_{ij}$ and the two-site mutual information
\begin{equation}
  \MI(i{:}j)
  = \SX(\rho_i) + \SX(\rho_j) - \SX(\rho_{ij}),
\end{equation}
with $\SX(\rho) = -\Tr(\rho \log \rho)$ the von Neumann entropy.
Averaging over all pairs with the same lattice distance $d=|i-j|$
yields $\MI(d)$, a one-dimensional profile of correlations.

Analytically, the exponential decay of correlation functions in the
gapped phase implies that $\MI(d)$ decays exponentially with $d$ on
scales larger than $\xiIGD$. Numerically, for finite chains (e.g.\
$N=8$) one observes a clear decrease of $\MI(d)$ with $d$, consistent
with this picture. This realizes informational locality (A9) on the
lattice: the information graph has short-range links of strength
set by $\xiIGD$.

For contiguous blocks $A$ of length $\ell$ in the ground state, the
entanglement entropy $S_A(\ell)$ saturates quickly as $\ell$ exceeds a
few correlation lengths. This is the one-dimensional version of an
area law: entanglement across the boundary between $A$ and its
complement remains bounded, in line with the T1 programme.

\section{Dynamical information spreading: light-cones}
\label{sec:tfim_dynamic}

To probe dynamical aspects of IGD, one can consider a global quench
from a simple product state, e.g.
\begin{equation}
  |\psi(0)\rangle = |\uparrow \uparrow \cdots \uparrow\rangle_z,
\end{equation}
and study the time-evolved state
\begin{equation}
  |\psi(t)\rangle = e^{-iHt} |\psi(0)\rangle.
\end{equation}
Two key diagnostics are:

\begin{itemize}
  \item the entanglement entropy $S_A(t)$ of half the chain;
  \item the distance-dependent mutual information $\MI(d,t)$ between
        sites at separation $d$.
\end{itemize}

For a small chain (e.g.\ $N=8$) one can compute these exactly by
diagonalizing $H$ and performing explicit partial traces. The numerical
results show:

\begin{itemize}
  \item $S_A(t)$ grows from zero at $t=0$, approximately linearly at
        early times, and then bends towards a saturation value set by
        the finite system size.
  \item The profile $\MI(d,t)$ exhibits an \emph{information
        light-cone}: for each fixed $d$, $\MI(d,t)$ remains small
        until $t \sim d/\vIGD$, then grows and eventually oscillates
        and saturates.
\end{itemize}

This behaviour is consistent with a Lieb--Robinson picture where
information propagation is limited by the lattice velocity
$\vIGD = v_{\max} = 1$. In IGD language, the emergent causal structure
on the information graph is a direct manifestation of A9 combined with
the core dynamics encoded in $J$ and $\Rop$.

\section{Role in the spacetime programme}
\label{sec:tfim_spacetime}

Within the broader IGD spacetime programme, the gapped TFIM serves as
a controlled lattice laboratory where one can:

\begin{itemize}
  \item test area-law statements (T1) in a fully solvable setting;
  \item study how coarse-graining of the information graph leads to
        effective low-dimensional geometries;
  \item benchmark numerical methods (exact diagonalization, MPS/DMRG)
        for more complex models.
\end{itemize}

The TFIM thus anchors the abstract axioms A0--A10 and the IGD core
equation in a simple, physically transparent model.
\end{document}
