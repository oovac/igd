\documentclass[11pt]{article}
\usepackage[margin=2.5cm]{geometry}
\usepackage[utf8]{inputenc}
\usepackage[T1]{fontenc}
\usepackage[english]{babel}

% Shared notation and symbols
% File: IGD_Notation_v1.tex
% Shared notation and symbol conventions for IGD papers.
%
% This file is intended to be \input in the preamble of each IGD LaTeX document.

\usepackage{amsmath,amssymb,amsfonts,amsthm}
\usepackage{bm}

% --- Sets, spaces, manifolds ---

% Hilbert space and state space
\newcommand{\Hilb}{\mathcal{H}}           % underlying Hilbert space
\newcommand{\States}{\mathcal{S}}         % space of physical states (density operators)
\newcommand{\Pure}{\mathcal{P}}           % manifold of pure states

% Parameter / model manifolds
\newcommand{\Mtheta}{\mathcal{M}_\theta}  % model (focus) manifold
\newcommand{\Mphys}{\mathcal{M}}          % generic state manifold

% Subsystems
\newcommand{\HA}{\Hilb_A}
\newcommand{\HB}{\Hilb_B}

% --- Geometry: metric, symplectic form, complex structure ---

\newcommand{\gij}{g_{ij}}                 % components of the information metric
\newcommand{\gmat}{g}                     % metric tensor
\newcommand{\om}{\omega}                  % symplectic 2-form
\newcommand{\Jc}{\mathcal{J}}             % (almost) complex structure on pure states
\newcommand{\Jop}{J}                      % antisymmetric operator implementing Hamiltonian flow
\newcommand{\Rop}{R}                      % symmetric operator implementing dissipative flow

% --- Energetic and entropic functionals ---

\newcommand{\Efun}{E}                     % energy functional on \Mtheta
\newcommand{\Sfun}{S}                     % entropy functional on \Mtheta

% --- Gradients and flows ---

\newcommand{\grad}{\nabla}                % gradient with respect to g
\newcommand{\dotth}{\dot{\theta}}        % time derivative of parameters
\newcommand{\ddt}{\frac{d}{dt}}          % total time derivative

% --- Information-theoretic quantities ---

\newcommand{\MI}{I}                       % mutual information
\newcommand{\SX}{S}                       % entropy (generic)
\newcommand{\KB}{k_{\mathrm{B}}}          % Boltzmann constant

\newcommand{\xiIGD}{\xi}                  % correlation / information length scale
\newcommand{\vIGD}{v_{\mathrm{IGD}}}      % information velocity (Lieb--Robinson-like)

% --- Misc ---

\newcommand{\Tr}{\mathrm{Tr}}
\newcommand{\id}{\mathbbm{1}}


\title{Information--Geometric Dynamics:\\
Core Equation and Spacetime Theorems (v3.0)}
\author{Ordo Vacui and Collaborator}
\date{\today}

\begin{document}
\maketitle

\begin{abstract}
This document formulates the core dynamical equation of
Information--Geometric Dynamics (IGD) and states three target
spacetime theorems (T1--T3). The aim is to make explicit how the
axioms A0--A10 support a GENERIC-like equation of motion for
effective states, and how locality, thermodynamics and complexity
enter the emergent spacetime picture. Detailed proofs and technical
results are deferred to future work; here we focus on clean
statements, assumptions and structural constraints.
\end{abstract}

\tableofcontents

\section{Overview}
\label{sec:overview}

IGD describes the evolution of \emph{effective} states
$\theta(t)$ on a model manifold $\Mtheta$ endowed with an
information metric $\gmat$, a symplectic structure $\om$ and a
complex structure $\Jc$, inherited from the underlying quantum
state space via the axioms A0--A10. The core equation of motion
takes the form
\begin{equation}
  \dot{\theta}
  = \Jop(\theta)\,\grad \Efun(\theta)
  + \Rop(\theta)\,\grad \Sfun(\theta),
  \label{eq:IGD_core}
\end{equation}
where $\Efun$ and $\Sfun$ are effective energy and entropy
functionals, $\Jop$ is antisymmetric (Hamiltonian flow) and
$\Rop$ is symmetric positive semi-definite (entropy-producing
gradient flow). This structure is analogous to the GENERIC
formalism in non-equilibrium thermodynamics, but here it is
explicitly anchored in quantum information geometry.

The second part of this document states three target theorems:
\begin{itemize}
  \item T1: an informational area law (and related holographic
        bounds) for entanglement;
  \item T2: an emergent Einstein-like equation as a thermodynamic
        relation for entanglement and energy flow;
  \item T3: a link between growth of computational complexity and
        geometric quantities in the emergent spacetime description.
\end{itemize}
These theorems are not assumed as axioms; they are goals of the
spacetime programme built on top of the IGD foundation.

\section{Core IGD dynamics}
\label{sec:core_dynamics}

\subsection{State space, focus manifolds and projections}
\label{subsec:state_focus}

Let $\States$ denote the global quantum state space (density
operators on a Hilbert space $\Hilb$), equipped with the Bures
information metric and the associated K\"ahler structure on pure
states, as per the axioms A0 and A10. A \emph{focus manifold}
$\Mtheta$ is a finite- or infinite-dimensional manifold of
effective states (macrostates, model parameters) together with
a map
\begin{equation}
  \pi: \States \to \Mtheta,
\end{equation}
interpreted as a coarse-graining or projection of the full
microscopic description. Different choices of focus realise
different effective theories; A8 requires IGD to be covariant
under changes of focus.

\subsection{Metric, symplectic and complex structures}
\label{subsec:metric_structures}

The pullback of the Bures metric from $\States$ and the induced
K\"ahler structure on suitable submanifolds of $\Mtheta$ provide
$\Mtheta$ with
\begin{itemize}
  \item an information metric $\gmat$,
  \item a symplectic form $\om$,
  \item an almost complex structure $\Jc$,
\end{itemize}
forming a K\"ahler-like triple in the regimes where the focus
manifold is compatible with the underlying quantum structure
(A0, A10). The metric $\gmat$ defines the gradient operator
$\grad$ and scalar products of tangent vectors, while $\om$ and
$\Jc$ encode the Hamiltonian flow structure associated with
reversible dynamics.

\subsection{Operators $J$ and $R$: structure and constraints}
\label{subsec:J_R_constraints}

The core IGD equation \eqref{eq:IGD_core} involves two
operators acting on tangent vectors:
\begin{itemize}
  \item $\Jop(\theta)$: an antisymmetric operator (with respect
        to $\gmat$) generating reversible flow;
  \item $\Rop(\theta)$: a symmetric positive semi-definite
        operator generating irreversible, entropy-producing flow.
\end{itemize}

More precisely, we impose the following structural conditions:
\begin{enumerate}
  \item \textbf{Antisymmetry of $\Jop$.} For any tangent vectors
        $X, Y$ at $\theta$,
        \begin{equation}
          \langle X, \Jop Y \rangle_g
          = - \langle \Jop X, Y \rangle_g,
        \end{equation}
        where $\langle \cdot,\cdot\rangle_g$ is the inner product
        defined by $\gmat$. This ensures that the flow generated
        by $\Jop \grad \Efun$ preserves the norm induced by
        $\gmat$ and is analogous to a Hamiltonian flow.
  \item \textbf{Symmetry and positivity of $\Rop$.} For any tangent
        vectors $X, Y$,
        \begin{equation}
          \langle X, \Rop Y \rangle_g
          = \langle \Rop X, Y \rangle_g,
        \end{equation}
        and
        \begin{equation}
          \langle X, \Rop X \rangle_g \ge 0.
        \end{equation}
        This ensures that the entropy production rate is
        non-negative.
  \item \textbf{Degeneracy conditions.} The reversible and
        irreversible parts satisfy orthogonality relations such
        as
        \begin{equation}
          \langle \grad \Efun, \Rop \grad \Sfun \rangle_g = 0,
          \qquad
          \langle \grad \Sfun, \Jop \grad \Efun \rangle_g = 0,
        \end{equation}
        so that energy change is governed by the reversible flow
        and entropy production by the irreversible flow, in line
        with the GENERIC philosophy.
\end{enumerate}

These conditions link directly back to the axioms:
A5 (processes as channels, data processing), A6 (Landauer and
thermodynamic cost), A8 (covariance under change of focus) and
A10 (geometric compatibility).

\subsection{Entropy production and time arrow}
\label{subsec:entropy_arrow}

Given the IGD equation \eqref{eq:IGD_core}, the rate of change
of the entropy functional is
\begin{equation}
  \dot{\Sfun}(\theta)
  = \langle \grad \Sfun, \dot{\theta} \rangle_g
  = \langle \grad \Sfun, \Jop \grad \Efun \rangle_g
  + \langle \grad \Sfun, \Rop \grad \Sfun \rangle_g.
\end{equation}
Under the degeneracy conditions above, the first term vanishes,
and positivity of $\Rop$ implies
\begin{equation}
  \dot{\Sfun}(\theta) \ge 0.
\end{equation}
Thus the entropy functional $\Sfun$ defines an arrow of time
for the irreversible part of the IGD flow, in harmony with the
thermodynamic arrow captured by A6.

\section{Spacetime theorems: statements and assumptions}
\label{sec:theorems}

We now state three target theorems of the IGD spacetime programme.
At this stage, the statements are meant to be precise enough to
guide mathematical work, but we do not claim to have complete
proofs. Instead we outline assumptions, regimes of validity and
the intended conclusions.

\subsection{T1: Informational area law and holographic bounds}
\label{subsec:T1}

\paragraph{Informal statement.}
Under suitable locality and gapped-spectrum assumptions, ground
states and low-lying states of IGD-compatible systems exhibit an
area law for entanglement entropy: the entanglement between a
region and its complement scales with the ``area'' of the boundary,
as measured in the emergent information geometry.

\paragraph{Assumptions (sketch).}
\begin{itemize}
  \item A0--A10 hold (quantum structure, focus manifolds, locality
        via A9, K\"ahler compatibility via A10).
  \item The system admits a decomposition into local subsystems
        arranged on an information graph whose coarse-grained
        limit defines an effective spatial manifold.
  \item The effective Hamiltonian (or generator of the reversible
        part) has a non-zero spectral gap above the ground state,
        and interactions are quasi-local in the information graph
        sense.
\end{itemize}

\paragraph{Expected conclusion.}
For ground states and a suitable class of low-energy states,
the entanglement entropy $S(A)$ of a region $A$ satisfies
\begin{equation}
  S(A) \le \alpha\, \mathrm{Area}(\partial A) + \cdots,
\end{equation}
where $\alpha$ is a constant (possibly related to fundamental
information scales) and $\mathrm{Area}(\partial A)$ is computed
in the emergent geometry induced by A9 and the IGD dynamics.
In discrete settings this reduces to a boundary-law scaling on
the information graph; in continuum limits it becomes an area
law reminiscent of holographic bounds.

\subsection{T2: Einstein-like equations from entanglement thermodynamics}
\label{subsec:T2}

\paragraph{Informal statement.}
In regimes where a smooth spacetime geometry emerges from the
information graph (A9), and where local regions are close to
thermal equilibrium, variations of entanglement entropy and
energy flux across local horizons satisfy a thermodynamic
relation whose geometric rewriting yields Einstein-like field
equations for the emergent metric.

\paragraph{Assumptions (sketch).}
\begin{itemize}
  \item A0--A10 hold, with a focus manifold $\Mtheta$ adapted to
        coarse-grained, approximately local observables.
  \item The information graph admits an IR limit described by a
        Lorentzian manifold with a notion of local Rindler horizons
        for accelerated observers.
  \item States of interest satisfy a \emph{local thermodynamic
        equilibrium} condition (LTE) so that an entanglement first
        law holds:
        \begin{equation}
          \delta S_A = \delta \langle K_A \rangle,
        \end{equation}
        where $K_A$ is a modular Hamiltonian associated with the
        region $A$.
\end{itemize}

\paragraph{Expected conclusion.}
By identifying the modular Hamiltonian with an effective energy
density integrated over the local horizon, and invoking the IGD
link between information flow and resource flow (A5, A6), one
expects to obtain a relation of the form
\begin{equation}
  \delta Q = T \,\delta S \;\;\Rightarrow\;\;
  G_{\mu\nu} + \Lambda g_{\mu\nu} = 8\pi G\, T_{\mu\nu},
\end{equation}
where $G_{\mu\nu}$ is the emergent Einstein tensor of the
coarse-grained geometry, $T_{\mu\nu}$ is an effective stress-energy
tensor and $G$ is an emergent gravitational coupling. The precise
form and constants depend on the microscopic details and the
choice of focus; the key point is that an Einstein-like geometry
arises as an \emph{equation of state} of entanglement, not as an
independent axiom.

\subsection{T3: Complexity growth and emergent geometry}
\label{subsec:T3}

\paragraph{Informal statement.}
For chaotic IGD systems with fast scrambling dynamics, the growth
of computational complexity along IGD trajectories is related to
geometric quantities (such as volumes or actions) in the emergent
spacetime picture. This provides a bridge between A7 (bounded
description complexity) and geometric diagnostics of chaos.

\paragraph{Assumptions (sketch).}
\begin{itemize}
  \item A0--A10 hold, with an explicit notion of computational
        complexity for states or unitaries (e.g.\ Nielsen
        geometry on the space of circuits or unitaries).
  \item The IGD flow $\theta(t)$ corresponds to physically
        implementable dynamics with a notion of simple vs
        complex directions in tangent space.
  \item The system is in a regime of strong chaos and fast
        scrambling, so that complexity grows approximately
        linearly in time up to very large scales.
\end{itemize}

\paragraph{Expected conclusion.}
There exists a functional $\mathcal{C}[\theta(t)]$ (complexity)
and a geometric quantity $V(t)$ or $\mathcal{A}(t)$ (e.g.\ volume
of a maximal slice or an on-shell action in an emergent bulk
geometry) such that, in an appropriate regime,
\begin{equation}
  \mathcal{C}(t) \propto V(t)
  \quad \text{or} \quad
  \mathcal{C}(t) \propto \mathcal{A}(t).
\end{equation}
The proportionality constants and the precise definitions of
$\mathcal{C}$ and $V$/$\mathcal{A}$ depend on the model, but the
structural claim is that complexity growth is not arbitrary: it
is tied to geometric features of the IGD-induced spacetime.

\section{Outlook}
\label{sec:outlook}

The present v3.0 document fixes the structure of the IGD core
equation and clarifies the intended form and assumptions of the
spacetime theorems T1--T3. The next steps in the IGD programme
are:
\begin{itemize}
  \item to work out concrete classes of $(\gmat,\om,\Jc,\Jop,\Rop)$
        compatible with A0--A10 and the constraints of this
        document;
  \item to establish T1--T3 rigorously for specific models
        (spin chains, tensor networks, holographic codes,
        SYK-like systems, etc.);
  \item to connect these results to phenomenological physics
        (quantum field theories, cosmology, black holes) via
        appropriate choices of focus manifolds and limits.
\end{itemize}

\end{document}
