\documentclass[11pt]{article}
\usepackage[margin=2.5cm]{geometry}
\usepackage[utf8]{inputenc}
\usepackage[T1]{fontenc}
\usepackage[english]{babel}

% File: IGD_Notation_v1.tex
% Shared notation and symbol conventions for IGD papers.
%
% This file is intended to be \input in the preamble of each IGD LaTeX document.

\usepackage{amsmath,amssymb,amsfonts,amsthm}
\usepackage{bm}

% --- Sets, spaces, manifolds ---

% Hilbert space and state space
\newcommand{\Hilb}{\mathcal{H}}           % underlying Hilbert space
\newcommand{\States}{\mathcal{S}}         % space of physical states (density operators)
\newcommand{\Pure}{\mathcal{P}}           % manifold of pure states

% Parameter / model manifolds
\newcommand{\Mtheta}{\mathcal{M}_\theta}  % model (focus) manifold
\newcommand{\Mphys}{\mathcal{M}}          % generic state manifold

% Subsystems
\newcommand{\HA}{\Hilb_A}
\newcommand{\HB}{\Hilb_B}

% --- Geometry: metric, symplectic form, complex structure ---

\newcommand{\gij}{g_{ij}}                 % components of the information metric
\newcommand{\gmat}{g}                     % metric tensor
\newcommand{\om}{\omega}                  % symplectic 2-form
\newcommand{\Jc}{\mathcal{J}}             % (almost) complex structure on pure states
\newcommand{\Jop}{J}                      % antisymmetric operator implementing Hamiltonian flow
\newcommand{\Rop}{R}                      % symmetric operator implementing dissipative flow

% --- Energetic and entropic functionals ---

\newcommand{\Efun}{E}                     % energy functional on \Mtheta
\newcommand{\Sfun}{S}                     % entropy functional on \Mtheta

% --- Gradients and flows ---

\newcommand{\grad}{\nabla}                % gradient with respect to g
\newcommand{\dotth}{\dot{\theta}}        % time derivative of parameters
\newcommand{\ddt}{\frac{d}{dt}}          % total time derivative

% --- Information-theoretic quantities ---

\newcommand{\MI}{I}                       % mutual information
\newcommand{\SX}{S}                       % entropy (generic)
\newcommand{\KB}{k_{\mathrm{B}}}          % Boltzmann constant

\newcommand{\xiIGD}{\xi}                  % correlation / information length scale
\newcommand{\vIGD}{v_{\mathrm{IGD}}}      % information velocity (Lieb--Robinson-like)

% --- Misc ---

\newcommand{\Tr}{\mathrm{Tr}}
\newcommand{\id}{\mathbbm{1}}


\title{IGD Applications: 1+1D Conformal Field Theories (v1.0)}
\author{Ordo Vacui and Collaborator}
\date{\today}

\begin{document}
\maketitle

\begin{abstract}
We sketch how Information--Geometric Dynamics (IGD) manifests in
1+1-dimensional conformal field theories (CFTs). In such theories
static entanglement entropies of intervals obey universal logarithmic
scaling laws, while global quenches produce linear growth and
saturation patterns reminiscent of information light-cones. We
interpret these results through the IGD lens, emphasizing the roles
of informational locality, effective information velocity and the
continuum limit of lattice models such as the TFIM.
\end{abstract}

\tableofcontents

\section{Motivation and setup}
\label{sec:cft_intro}

In 1+1D CFTs, the ground state and low-lying excitations exhibit
universal entanglement patterns governed by the central charge $c$.
These patterns can be seen as the continuum counterpart of the area
laws and light-cones observed in lattice models. From the IGD
perspective, CFTs provide an analytically tractable limit of the
information geometry associated with critical spin chains and related
systems.

We consider a CFT on either the infinite line or a circle of
circumference $L$, and focus on the entanglement entropy $S_A$ of a
spatial interval $A$ of length $\ell$. We set the characteristic
velocity of excitations to $v=1$ (this $v$ should not be confused with
$\vIGD$; in this context they coincide by construction).

\section{Static entanglement in 1+1D CFT}
\label{sec:cft_static}

For the ground state of a 1+1D CFT on the infinite line, the
entanglement entropy of an interval of length $\ell$ takes the form
\begin{equation}
  S_A(\ell) = \frac{c}{3} \log\!\left(\frac{\ell}{a}\right) + s_0,
\end{equation}
where $c$ is the central charge, $a$ is a short-distance cutoff and
$s_0$ is a non-universal constant. On a circle of circumference $L$
one has
\begin{equation}
  S_A(\ell)
  = \frac{c}{3} \log\!\left(
      \frac{L}{\pi a}
      \sin\frac{\pi \ell}{L}
    \right) + s_0.
\end{equation}

From the IGD viewpoint, the absence of a finite correlation length
($\xiIGD \to \infty$) at criticality is reflected in the logarithmic
dependence on $\ell$. Nevertheless, the entanglement is still controlled
by a ``boundary'' (the endpoints of the interval), and the central
charge $c$ acts as a measure of the effective number of degrees of
freedom (information channels) per unit length.

\section{Global quenches and entanglement growth}
\label{sec:cft_quench}

A global quantum quench in a 1+1D CFT can be modeled by preparing a
short-range entangled state $|\psi_0\rangle$ and then evolving it with
the CFT Hamiltonian. A simple picture in terms of pairs of
entangled quasiparticles leads to the following qualitative behaviour
for the entanglement entropy $S_A(t)$ of an interval $A$ of length
$\ell$:
\begin{equation}
  S_A(t) \simeq
  \begin{cases}
    \dfrac{c}{3} \log\!\left(\dfrac{2 v t}{a}\right) + \text{const},
      & t \lesssim \ell/(2v),\\[1ex]
    \dfrac{c}{3} \log\!\left(\dfrac{\ell}{a}\right) + \text{const},
      & t \gtrsim \ell/(2v).
  \end{cases}
\end{equation}
At early times, entangled pairs produced near the initial ``quench
surface'' propagate ballistically with speed $v$ in opposite directions
and contribute to the entanglement of $A$ once both members of a pair
enter $A$ and its complement. At late times, the entanglement saturates
to a value set by the interval size.

In addition, correlation functions and mutual information between
spatially separated regions display light-cone structures: correlations
become appreciable only when signals traveling at speed $v$ have had
time to connect the regions.

\section{IGD interpretation}
\label{sec:cft_igd}

The CFT setting realizes IGD concepts in a particularly sharp way:

\begin{itemize}
  \item \textbf{Information velocity.} The speed $v$ of quasiparticles
        plays the role of an information velocity $\vIGD$ in the
        continuum limit. Light-cone effects in entanglement and
        correlations directly reflect the causal structure of the
        emergent spacetime.
  \item \textbf{Critical geometry.} The logarithmic scaling of
        $S_A(\ell)$ indicates the absence of a finite correlation
        length $\xiIGD$, replacing the simple exponential decay of
        correlations with scale invariance. The underlying IGD geometry
        is controlled by conformal symmetry rather than by a finite
        length scale.
  \item \textbf{Link to lattice models.} Critical lattice models such
        as the TFIM at its quantum critical point flow in the infrared
        to CFTs. The IGD structures identified on the lattice thus have
        a well-defined continuum limit, with the central charge $c$
        and the velocity $v$ encapsulating key information-geometric
        properties.
\end{itemize}

CFTs thereby serve as a bridge between discrete IGD realizations
(spin chains, tensor networks) and the continuum spacetime picture
explored in the T1--T3 spacetime programme.

\section{Outlook}
\label{sec:cft_outlook}

A more detailed IGD treatment of 1+1D CFTs would involve explicit
information metrics on spaces of CFT states, modular Hamiltonians for
intervals and their relation to emergent geometry. The present v1.0
note is intended as a concise anchor linking known CFT entanglement
results to the broader IGD framework.
\end{document}
