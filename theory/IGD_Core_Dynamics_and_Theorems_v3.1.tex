\documentclass[11pt]{article}
\usepackage[margin=2.5cm]{geometry}
\usepackage[utf8]{inputenc}
\usepackage[T1]{fontenc}
\usepackage{amsmath,amssymb,amsfonts}
\usepackage{bm}
\usepackage{hyperref}

\title{Information--Geometric Dynamics:\\
Core Dynamics and Spacetime Theorems (v3.1)}
\author{Ordo Vacui and Collaborator}
\date{\today}

\begin{document}
\maketitle

\begin{abstract}
This note collects the core dynamical structure of Information--Geometric
Dynamics (IGD) and summarises the three spacetime theorems (T1--T3)
together with their concrete model realisations implemented in the
multimodel IGD simulator. It is intended as an overview document,
linking the axiomatic foundation to the TFIM (gapped), CFT (critical)
and SYK$_4$ (chaotic) testbeds.
\end{abstract}

\section{IGD axioms and core dynamics}

We assume the full axiomatic basis A0--A10 has been introduced in the
separate foundation text (\emph{IGD\_Axiomatic\_Foundation\_v2.3},
\emph{IGD\_Axiomatic\_Foundation\_v3.0}, etc.). Here we only recall the
minimal ingredients needed to state the spacetime theorems.

\begin{itemize}
  \item \textbf{Quantum structure (A0):} the state space is that of
        density operators on a complex Hilbert space, with purification
        and local tomography ensuring the standard quantum framework
        (Hilbert space, tensor product composition, entanglement).
  \item \textbf{Information and resources (A1--A7):} information is
        treated as a physical resource subject to constraints such as
        Landauer's principle and bounded computational complexity.
  \item \textbf{Focus and geometry (A8--A10):} physical descriptions
        live on focus manifolds $\mathcal{M}_\theta$ embedded in the
        full state space, equipped with an information metric (Bures),
        a compatible symplectic form and a complex structure. The
        reversible dynamics is a K\"ahler isometry and the irreversible
        dynamics is a gradient flow of entropy with respect to the same
        metric.
\end{itemize}

The generic IGD equation of motion on a focus manifold $\mathcal{M}_\theta$
has the GENERIC form
\begin{equation}
  \dot{\theta} = J(\theta)\,\nabla E(\theta) + R(\theta)\,\nabla S(\theta),
\end{equation}
where $J$ encodes the reversible (Hamiltonian) part and $R$ the
irreversible (entropic) part. The spacetime theorems T1--T3 describe
emergent geometric behaviour of this dynamics in different regimes.

\section{Spacetime theorems T1--T3: overview}

\subsection{T1: informational area laws in gapped regimes}

T1 addresses entanglement area laws in gapped phases with short-range
correlations. Informally, for local gapped Hamiltonians satisfying the
IGD locality and clustering assumptions, ground states obey an
entanglement area law: the entropy of a region scales with the
``area'' of its boundary rather than its volume.

In one spatial dimension this reduces to the statement that the
entropy $S_A(\ell)$ of a contiguous block of length $\ell$ in a
gapped spin chain is bounded by a constant independent of $\ell$,
once the block is large compared to the correlation length
$\xi_{\mathrm{IGD}}$ extracted from the information metric.

A detailed IGD formulation of T1 and its relation to known area-law
theorems in one dimension is given in
\emph{IGD\_T1\_AreaLaw\_SpinChains\_v1.0.tex}. The transverse-field
Ising model (TFIM) in its gapped phase provides a canonical example;
its IGD analysis and numerical simulations are documented in
\emph{IGD\_Lattice\_SpinChains\_v1.0.tex}.

\subsection{T2: entanglement thermodynamics and Einstein-like equations}

T2 focuses on the emergence of Einstein-like field equations from
entanglement thermodynamics in appropriate near-equilibrium regimes.
The guiding idea is that for small perturbations of the vacuum of a
local quantum field theory (or a suitable many-body system), changes
in entanglement entropy of local regions obey a first-law relation
$\delta S = \delta \langle K \rangle$, where $K$ is the modular
Hamiltonian. Combined with an identification of $\delta \langle K
\rangle$ with energy flux through local causal horizons, this leads
to equations of motion for an emergent spacetime metric that resemble
Einstein's equations.

In the IGD setting, T2 asserts that under the axioms A0--A10 and
additional assumptions of local thermodynamic equilibrium (LTE) and
smooth coarse-graining (RG flow to an infrared fixed point), there
exists an emergent geometric description in which the entanglement
structure of states determines an effective metric obeying
Einstein-like equations in a suitable linearised regime.

A full technical implementation of T2 is beyond the scope of the
present v3.1 note; the CFT and critical spin-chain sectors in the
simulator provide the natural laboratory for developing this
entanglement-thermodynamic perspective.

\subsection{T3: chaos, complexity and emergent geometry}

T3 addresses chaotic regimes where the dominant feature of the IGD
flow is rapid growth of complexity and scrambling of information. Its
informal content is:

\begin{itemize}
  \item there exists a complexity functional $\mathcal{C}[\rho(t)]$
        associated with information--geometric lengths of trajectories
        in an appropriate metric on state or unitary space;
  \item operational complexity proxies such as half-system entanglement
        $S_A(t)$ and out-of-time-order correlator (OTOC) diagnostics
        $C(t)$ track the growth and saturation of $\mathcal{C}[\rho(t)]$;
  \item this complexity growth is encoded in emergent geometric
        quantities (volumes, actions, lengths) associated with the IGD
        flow, in the spirit of complexity--volume or complexity--action
        ideas, but formulated in a fully information--geometric
        framework.
\end{itemize}

A concrete realisation of the T3 playground is provided by a small
Majorana SYK$_4$--like model implemented on a few qubits, with
entanglement and OTOC diagnostics computed exactly. The corresponding
phenomenology and its IGD interpretation are described in
\emph{IGD\_Chaos\_Complexity\_T3\_v1.0.tex}.

\section{Model sectors and simulator mapping}

The multimodel IGD simulator implements three main sectors that map
directly onto the T1--T3 structure:

\begin{itemize}
  \item \textbf{Gapped sector (TFIM/T1):} one-dimensional gapped
        spin chains, with explicit computation of entanglement
        entropies $S_A(\ell)$, mutual information $I(d)$, dynamical
        entanglement growth $S_A(t)$ after quenches, and information
        light-cones $I(d,t)$.
  \item \textbf{Critical sector (CFT/critical TFIM):} an effective
        1+1D CFT description of critical spin chains, with logarithmic
        entanglement scaling and quench dynamics, plus a comparison
        between critical TFIM and CFT predictions.
  \item \textbf{Chaotic sector (SYK$_4$/T3):} a small Majorana SYK$_4$
        toy model with exact time evolution, half-system entanglement
        growth $S_A(t)$ and OTOC-based scrambling diagnostics $C(t)$.
\end{itemize}

The present v3.1 core note should be read together with the detailed
model-specific texts and the simulator documentation, which show in
explicit numerical form how the abstract spacetime theorems T1--T3
manifest in concrete quantum many-body systems.
\end{document}
